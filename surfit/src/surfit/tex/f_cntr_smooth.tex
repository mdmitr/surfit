\documentclass{article}

\usepackage[dvips]{graphicx}
%\usepackage[cp1251]{inputenc}
%\usepackage[russian]{babel}
\usepackage{amsmath}
\usepackage{indentfirst}  % ��� ������� ������
\usepackage{graphicx}
\usepackage{float}


\renewcommand{\labelenumi}{$\bullet$}

\begin{document}

Problem: we need to approximate contour by adjusting cell values. 
Consider Figure \ref{pic1}. Solid straight lines show cells borders, dot lines connect
cells centers (black squares), thick black curves shows contours projection on XoY plane. 
Red points on the figure are intersection points of curves with dot lines. 

\begin{figure}[!htbp]                  
\begin{center}
\includegraphics*[]{matr_sect2.eps}
\end{center}
\caption{Contour crosses virtual lines between grid nodes.} \label{pic1}
\end{figure}

Consider the case then the contour crosses one of the cell borders (see Figure \ref{pic2}). 
On the figure $\overline{p}_0$ and $\overline{p}_1$ are cells centers coordinates (X or Y), 
$u_0$ and $u_1$ are adjustable values for these cells. Red circles are points where contours cross
line (dot line on Figure \ref{pic1}) between cells centers. 
$p_i$ - are points coordinates (X or Y), $z_i$ - Z coordinates.

\begin{figure*}[!htbp]
\begin{center}
\includegraphics*[]{matr_sect.eps}
\end{center}
\caption{Contour crosses line between two grid nodes.} \label{pic2}
\end{figure*}

For solving our task we write the following functional:

\begin{equation}\label{phi_0_new_terms}
\Phi = \sum\limits_{i=1}^N \left( f(p_i) - z_i \right)^2
\rightarrow \min, 
\end{equation}
where
\begin{equation}\label{f_p}
 f(p) = u_0 + (u_1-u_0)
\frac{p-\overline{p}_0}{\overline{p}_1-\overline{p}_0}.
\end{equation}

Speaking simply, we are trying to approximate $z_i$ values with straight line.
To solve our problem we should write this functional for each pair of 
neighbour cells, if at least one of contours lays between them.


With respect to \eqref{f_p} we can rewrite \eqref{phi_0_new_terms}:
\begin{equation}\label{phi}
\Phi = \sum\limits_{i=1}^N \left( u_0 + (u_1-u_0)
\frac{p_i-\overline{p}_0}{\overline{p}_1-\overline{p}_0} - z_i
\right)^2 \rightarrow \min.
\end{equation}
To find minimum of this functional we should solve the system of
linear equations:
$$
\frac{ \partial \Phi } { \partial u_0 } =
%
2 \sum_i \left( u_0 + (u_1-u_0)
\frac{p_i-\overline{p}_0}{\overline{p}_1-\overline{p}_0} - y_i
\right)
%
\left( 1 -
\frac{p_i-\overline{p}_0}{\overline{p}_1-\overline{p}_0} \right) =
$$
$$
= 2 \sum_i \left( \left( 1 -
\frac{p_i-\overline{p}_0}{\overline{p}_1-\overline{p}_0} \right)
u_0 + \frac{p_i-\overline{p}_0}{\overline{p}_1-\overline{p}_0} u_1
- y_i \right) \left( 1 -
\frac{p_i-\overline{p}_0}{\overline{p}_1-\overline{p}_0} \right) =
$$
$$
= 2 u_0 \sum_i \left( 1 -
\frac{p_i-\overline{p}_0}{\overline{p}_1-\overline{p}_0} \right)^2
%
+ 2 u_1 \sum_i \left( 1 -
\frac{p_i-\overline{p}_0}{\overline{p}_1-\overline{p}_0} \right)
\frac{p_i-\overline{p}_0}{\overline{p}_1-\overline{p}_0}
%
- 2 \sum_i y_i \left( 1 -
\frac{p_i-\overline{p}_0}{\overline{p}_1-\overline{p}_0} \right)
=0
$$

%
%
%
%
%

$$
\frac{ \partial \Phi } { \partial u_1 } =
%
2 \sum_i \left( u_0 + (u_1-u_0)
\frac{p_i-\overline{p}_0}{\overline{p}_1-\overline{p}_0} - y_i
\right)
%
\frac{p_i-\overline{p}_0}{\overline{p}_1-\overline{p}_0} =
$$
$$
= 2 \sum_i \left( \left( 1 -
\frac{p_i-\overline{p}_0}{\overline{p}_1-\overline{p}_0} \right)
u_0 + \frac{p_i-\overline{p}_0}{\overline{p}_1-\overline{p}_0} u_1
- y_i \right)
%
\frac{p_i-\overline{p}_0}{\overline{p}_1-\overline{p}_0} =
$$
$$
= 2 u_0 \sum_i \left( 1 -
\frac{p_i-\overline{p}_0}{\overline{p}_1-\overline{p}_0} \right)
\frac{p_i-\overline{p}_0}{\overline{p}_1-\overline{p}_0}
%
+ 2 u_1 \left(
\frac{p_i-\overline{p}_0}{\overline{p}_1-\overline{p}_0} \right)^2
%
- 2 \sum_i y_i
\frac{p_i-\overline{p}_0}{\overline{p}_1-\overline{p}_0} = 0
$$

$$
\begin{cases}
u_0 \underbrace{\sum\limits_i \left( 1 -
\frac{p_i-\overline{p}_0}{\overline{p}_1-\overline{p}_0}
\right)^2}_{sum1}
%
+ \, u_1 \underbrace{\sum\limits_i \left( 1 -
\frac{p_i-\overline{p}_0}{\overline{p}_1-\overline{p}_0} \right)
\frac{p_i-\overline{p}_0}{\overline{p}_1-\overline{p}_0}}_{sum2}
%
= \, \underbrace{\sum\limits_i y_i \left( 1 -
\frac{p_i-\overline{p}_0}{\overline{p}_1-\overline{p}_0}
\right)}_{sum4}
\\
u_0 \underbrace{\sum\limits_i \left( 1 -
\frac{p_i-\overline{p}_0}{\overline{p}_1-\overline{p}_0} \right)
\frac{p_i-\overline{p}_0}{\overline{p}_1-\overline{p}_0}}_{sum2}
%
+ u_1 \, \underbrace{\sum\limits_i \left(
\frac{p_i-\overline{p}_0}{\overline{p}_1-\overline{p}_0}
\right)^2}_{sum3}
%
= \, \underbrace{\sum\limits_i y_i
\frac{p_i-\overline{p}_0}{\overline{p}_1-\overline{p}_0}}_{sum5}
\end{cases}
$$

System of equations for $u_0$:
$$
\begin{cases}
u_0 \underbrace{\sum\limits_i \left( 1 -
\frac{p_i-\overline{p}_0}{\overline{p}_1-\overline{p}_0}
\right)^2}_{sum1}
%
+ \, u_1 \underbrace{\sum\limits_i \left( 1 -
\frac{p_i-\overline{p}_0}{\overline{p}_1-\overline{p}_0} \right)
\frac{p_i-\overline{p}_0}{\overline{p}_1-\overline{p}_0}}_{sum2}
%
= \underbrace{\sum\limits_i y_i \left( 1 -
\frac{p_i-\overline{p}_0}{\overline{p}_1-\overline{p}_0}
\right)}_{sum4}
\\
u_{-1} \underbrace{\sum\limits_i \left( 1 - \frac {
p_i-\overline{p}_{-1}}{\overline{p}_0-\overline{p}_{-1}} \right)
\frac{p_i-\overline{p}_{-1}}{\overline{p}_0-\overline{p}_{-1}}
}_{sum2}
%
+ \, u_0 \underbrace{\sum\limits_i \left(
\frac{p_i-\overline{p}_{-1}}{\overline{p}_0-\overline{p}_{-1}}
\right)^2}_{sum3}
%
= \underbrace{\sum\limits_i y_i
\frac{p_i-\overline{p}_{-1}}{\overline{p}_0-\overline{p}_{-1}}}_{sum5}
\end{cases}
$$

\end{document}
